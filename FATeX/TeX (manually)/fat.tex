% 
% Автор: Гончаренко Дмитрий, GlowByte Consulting
% 
\documentclass[12pt,a4paper]{extarticle}
\usepackage{cmap}
\usepackage[utf8x]{inputenc}
\usepackage[T2A]{fontenc}
\usepackage[russian,english]{babel}
\usepackage[hidelinks]{hyperref}  
\usepackage{amsthm}
\usepackage{listings,lstautogobble} 
\usepackage{amsmath}
\usepackage{amsfonts}
\usepackage{amssymb}
\usepackage{xcolor,colortbl}
\usepackage{graphicx}
\usepackage{subcaption}
\usepackage{tabularx}
\usepackage{fullpage}
\usepackage{fancyhdr}
\usepackage[nottoc,numbib]{tocbibind}

% -------------------------------------------------------

 %----------------------------------------------------------------------------------------
 %   INFORMATION SECTION
 %----------------------------------------------------------------------------------------

 \newcommand{\name}{Гончаренко Дмитрий Александрович} % ФИО
 
 \newcommand{\ftitle}{Что сделано} % 1. Что сделано

 \newcommand{\system}{SAS RTDM} % 2. Затрагиваемые ИТ системы

 \newcommand{\whattest}{Что тестировано} % 3. Предмет тестирования/FAT

 \newcommand{\whotest}{Гончаренко Д.А.} % 4. Участники тестирования/FAT

 \newcommand{\role}{Технолог} % 4. Участники тестирования/FAT - Роль

 \newcommand{\goal}{Зачем тестировано} % 5. Цель тестирования/FAT

 \newcommand{\env}{SAS RTDM} % 6. Среда проведения тестирования/FAT

 \newcommand{\result}{Приложение работоспособно и работает в соответствие с ожиданиями} % 7. Результат проверки тестирования/FAT


% -------------------------------------------------------

\pagestyle{fancy}
\renewcommand{\headrulewidth}{0pt}
\newenvironment{textbox}[1][|X|]{
    \vspace{1pt}
    \centering
    \tabularx{\textwidth}{#1}
        \hline 
    }
    { 
        \\\hline
    \endtabularx\\
    \vspace{4.5pt}
}


\begin{document}
\normalsize
\selectlanguage{russian}
\thispagestyle{empty}

    \setlength{\headsep}{-2cm}
    \begin{figure}[htp]
        \centering
        \includegraphics[width=0.3\textwidth]{rsb}
    \end{figure}
    \begin{center}
        \color{blue}{\textbf{
            \Huge{\underline{FAT}\textbackslash UAT Report}
        }}
    \end{center}

    \begin{textbox} \\
        1. \ftitle \\
    \end{textbox}

    \noindent\textbf{2. Затрагиваемые ИТ системы}

    \begin{textbox}
        \system
    \end{textbox}

    \noindent\textbf{3. Предмет тестирования/FAT}

    \begin{textbox}
        \whattest
    \end{textbox}

    \noindent\textbf{4. Участники тестирования/FAT}

    \begin{textbox}[|X|X|X|]
            \textbf{Имя}    & \textbf{Подразделение} & \textbf{Роль} \\ \hline
            \whotest        & ДРР                    & \role  
    \end{textbox}

    \noindent\textbf{5. Цель тестирования/FAT}
    
    \begin{textbox}
        \goal
    \end{textbox}

    \noindent\textbf{6. Среда проведения тестирования/FAT}
    
    \begin{textbox}
        \system
    \end{textbox}

    \noindent\textbf{7. Результат проверки тестирования/FAT}
    
    \begin{textbox}
        \result
    \end{textbox}

    \noindent\textbf{8. Заключение по результату проверки тестирования/FAT}
    
    \begin{textbox}
        \footnotesize{\emph{В процессе проведения FAT установлено, что функциональность ПО полностью соответствует предъявленным требованиям. Установку доработок в СЕРТ среду считаем возможным.}}
    \end{textbox}

    \noindent\textbf{9. Выявленные замечания по результату проверки тестирования/FAT}
    
    \begin{textbox}
        
    \end{textbox}

    \noindent\textbf{10. Финальное решение по результату проверки тестирования/FAT}
    
    \begin{textbox}[|l|X|]
        \textbf{Test\textbackslash FAT статус} & \textbf{Пройдено} \\ \hline
        \textbf{Комментарий} & \\ &
    \end{textbox}

    \vspace{-7pt}
    
    \begin{textbox}[|l|X|p{0.2\textwidth}|]
        Дата    & Фамилия, Имя Бизнес представителя & Подпись \\ \hline
        \today        & \name    & 
    \end{textbox}

    \vspace{-1.5cm}
    \hspace{12.5cm}\includegraphics[width=70pt]{sign}  


\end{document} 